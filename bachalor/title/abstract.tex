\begin{abstract}
Алгоритмы управления очередью, применяемые в сетевых маршрутизаторах, 
выполняют ключевую функцию в обеспечении высокого уровня качества обслуживания (Quality of Service, QoS), 
что в свою очередь способствует эффективному распределению сетевых ресурсов и удовлетворению требований пользователей к скорости и надежности передачи данных. 
В рамках выпускной квалификационной работы осуществляется анализ производительности алгоритма Random Early Detection (RED) и его модификаций через комплексное моделирование. 
Данный анализ включает в себя сравнение с другими дисциплинами управления очередями с использованием критериев, таких как размер очереди, задержка, 
вариативность задержки и изменение размера окна протокола TCP типов reno, vegas. Применение моделирования на базе симулятора NS-2, включающего модификацию исходного кода, и реализация на практике с использованием Mininet, а также инструментов iperf3, tc, netem, позволяют точно оценить производительность и эффективность различных настроек RED. 
Анализ результатов, полученных с помощью программы Gnuplot, 
демонстрирует, что алгоритмы семейства RED обеспечивают значительные преимущества по сравнению с традиционным механизмом Drop Tail, особенно в аспектах управления задержкой и ее вариативностью, а также снижения частоты отбрасывания пакетов. 
Эти выводы подкрепляются количественными данными и графическими иллюстрациями, что делает исследование актуальным для разработчиков сетевого оборудования, стремящихся оптимизировать процессы управления трафиком.
\end{abstract}

\makeabstract  
  
%%% Local Variables: 
%%% mode: latex
%%% coding: utf-8-unix
%%% End: 
