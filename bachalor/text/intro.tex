\chapter*{Введение}
\addcontentsline{toc}{chapter}{Введение}

Данное исследование посвящено анализу и сравнению алгоритмов управления очередью из семейства RED, реализованных с использованием программных средств NS-2 и Mininet. Центральная задача работы - исследовать принципы работы и эффективность алгоритмов RED через моделирование их поведения в различных сетевых условиях. В рамках работы предпринята попытка имтационного и натурного моделирования поведения сети при использовании алгоритмов RED, а также проведено сравнение их производительности с целью выявления наиболее эффективных настроек для обеспечения качественной и надежной передачи данных. Результаты моделирования предназначены для определения оптимальных параметров алгоритмов управления очередью, способствующих улучшению общей производительности сетевых систем.

\section*{Актуальность темы}

Актуальность данного исследования заключается в анализе механизма активного управления очередью RED, который вносит вклад в оптимизацию распределения ресурсов сети и обеспечивает соответствие требованиям пользователей по скорости и надежности передачи данных.

\section*{Цель работы:}

Целью данной выпускной квалификационной работы является исследование и анализ алгоритмов семейства RED для активного управления очередью в сетевых системах, а также оценка их эффективности с использованием различных инструментов моделирования

\section*{Структура работы}
%\section*{Краткое содержание выпускной работы}
Данная работа состоит из введения, трех разделов, заключения, списка используемой литературы и приложенией. Во введении мною приведено краткое описание работы, также обусловлена ее актуальность, поставлена цель и сформулированы задачи выпускной квалификационной работы.

В первом разделе работы рассмотрены средства моделирования сетей и принципы их работы.

Во втором разделе работы приведен обзор алгоритмов семейства RED и её реализации в итационной модели NS-2 и натурной модели в Mininet.

В третьем разделе выпускной квалификационной работы подробно описаны функции, отвечающие за
реализацию моделей. Показаны результаты моделирования, выведены необходимые графики функций и сделаны выводы об эффективности алгоритма с помощью двух средств моделирования.

В заключении подведены общие итоги работы, изложены основные выводы.


%%% Local Variables:
%%% mode: latex
%%% coding: utf-8-unix
%%% TeX-master: "../default"
%%% End:
