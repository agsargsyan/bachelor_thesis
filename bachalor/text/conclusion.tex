\chapter*{Заключение}

В ходе выполнения выпускной квалификационной работы был дан обзор
основных алгоритмов семейства RED. Был рассмотрены
средства моделирования NS-2 и Mininet и изучены существующие в них
инструменты для моделирования сетей передачи данных. 

Разработан программный комплекс, реализующий имитационной и натурной
моделей заданной топологии ~\cite{red-git}. Для визуализации результатов,
а также для оценки работы было использована программа GNUPLOT. Были описанны
структура и все функции, из которых состоит программный комплекс.   

Таким образом были выполнены задачи, сформулированные перед началом
работы:

\begin{enumerate}
\item Разобрано архитектуры средств моделирования и составлены алгоритмы
работы в данных симуляторах.
\item Разработан программный комплекс, позволяющий моделировать
разные топологии сети с дисциплиной RED.
\item Проведён анализ результатов моделирования сети с управлением
  очередями на маршрутизаторе и сделан вывод об эффективности работы
  алгоритма RED при запуске в разных средах.
\end{enumerate}


\clearpage


%%% Local Variables:
%%% mode: latex
%%% coding: utf-8-unix
%%% TeX-master: "../default"
%%% End:
