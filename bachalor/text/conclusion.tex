\chapter*{Заключение}
\addcontentsline{toc}{chapter}{Заключение}

Текст.

В работе было рассмотрено:
\begin{enumerate}
\item Принципы работы тонких клиентов, различные способы организации
  системы тонких клиентов 
\item Сделан обзор продуктов компании NX NoMachine, а также проекта
  FreeNX, созданного на основе открытых библиотек NX, выделены их
  преимущества
\item Произведен сравнительный анализ стоимости различных конфигураций
  дисплейных классов, сделан вывод в пользу класса на основе Х-терминалов. 
\item Произведено тестовое подключение компьютера с установленным на
  нем клиентом NX к FreeNX серверу, а также запуск на нем приложений с
  оценкой скорости их работы. Скорость работы оказалась вполне
  приемлемой.
\end{enumerate}
Итог: разработанный нами метод развертывания системы Х-терминалов
рекомендуется к применению в государственных и коммерческих учреждениях
ввиду обеспечиваемого им снижения затрат на организацию и администрирование. 


\clearpage



%%% Local Variables:
%%% mode: latex
%%% coding: utf-8-unix
%%% TeX-master: "../default"
%%% End: