\chapter*{Заключение}
\addcontentsline{toc}{chapter}{Заключение}

В ходе выполнения выпускной квалификационной работы был дан обзор используемых средств моделирования, типов дисциплины управлением очередью RED, а также рассмотрен алгоритм работы дисциплин данного семейства.
Разработан единый программный модуль, реализующий алгоритм работы RED в NS-2 и Mininet. Были задействованы стандартные классы, а также
средство Mininet, позволяющее строить графики по полученным данным.
Таким образом были выполнены задачи, сформулированные перед началом работы

В работе было рассмотрено:
\begin{enumerate}
\item Принципы работы таких средств моделирования сетей, как NS-2 и Mininet 
\item Сделан обзор алгоритма работы дисциплины управления очередью RED и некоторых его модификаций, а также
  разработаны их реализация в имтационной модели в NS-2.
\item Произведен сравнительный анализ результатов при имитационном и натурном моделировании сетис учетом одинаковых метрик. 
\end{enumerate}



\clearpage



%%% Local Variables:
%%% mode: latex
%%% coding: utf-8-unix
%%% TeX-master: "../default"
%%% End:
