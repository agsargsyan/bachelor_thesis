
% Первая глава работы 
\chapter{Средства моделирования сетей передачи данных}
\label{chap1}

В данном разделе представим краткий обзор используемых в работе средств моделирования сетей
передачи данных.

\section{NS-2}
\label{chap1:sec1}

NS-2 — это программное средство моделирования
сетей, использующееся для исследования и анализа поведения
компьютерных сетей.  Запуск имитационной модели в данной среде
позволяет анализировать различные протоколы и алгоритмы сетевой связи.

NS-2 состоит из двух частей: компонент, реализованный на языке C++, 
требующий перекомпиляции при модификации, и часть, написанная на языке Objective Tcl,
не нуждающаяся в компиляции. Обе части имеют пересекающиеся классы, называемые в 
терминах NS-2 компилируемой и интерпретируемой иерархиями соответственно. 
Взаимодействие между данными частями регулируется спецификацией, обеспечивающей вызовы 
методов из одной части в другую и обратно. Компилируемая часть предназначена для обработки задач, 
требующих высокой производительности, в то время как интерпретируемая часть управляет 
моделированием и манипулированием объектами. Данный подход позволяет легко создавать модели сетей 
с использованием Tcl, в то время как для изменения в компилируемой части требуется модификация 
кода на C++ и последующей перекомпиляция. Данная архитектура обеспечивает гибкость и расширяемость 
средства моделирования, например возможность самостоятельной реализации алгоритма новых дисциплин управления очередью. 

NS-2 включает в себя богатую библиотеку классов, которые обеспечивают моделирование широкого спектра сетевых 
элементов и поведений. Вот некоторые из ключевых элементов, которые можно моделировать с помощью NS-2

\begin{itemize}
\item Узлы сети: Виртуальные представления оконечных устройств, таких как компьютеры, мобильные устройства или серверы. 
Узлы могут генерировать, получать и перенаправлять данные в сети.
\item Маршрутизаторы: Элементы, отвечающие за перенаправление пакетов данных между узлами в сети. 
Маршрутизаторы используют различные алгоритмы маршрутизации для определения оптимальных путей передачи данных.
\item Каналы связи: Моделируют физические и виртуальные пути передачи данных между узлами. 
Каналы могут иметь различную пропускную способность, задержку и уровень ошибок, что позволяет исследовать влияние различных сетевых условий на производительность сети.
\item Протоколы передачи данных: Поддержка множества стандартных и экспериментальных протоколов для моделирования поведения сети. 
Это включает в себя протоколы транспортного уровня, такие как TCP и UDP, протоколы маршрутизации, такие как OSPF и BGP, а также протоколы прикладного уровня.
\end{itemize}


Для создания моделей сети определяются характеристики и
параметры каждого элемента сети: пропускная способность канала,
задержки, вероятность потери пакетов и другие. После завершения
симуляции NS-2 предоставляет мощные инструменты анализа результатов,
включая возможность визуализации данных посредством программы NAM,  
статистический анализ и сравнение результатов экспериментов, что позволяет 
изучать и оценивать производительность различных протоколов и алгоритмов 
в различных сценариях сети ~\cite{NS2-1, NS2-2}.

Алгоритм работы в NS-2 включает в себя несколько ключевых шагов:
\begin{enumerate}
\item Формирование структуры сетевого взаимодействия: процесс конструирования структуры сетевого взаимодействия включающая в себя разработку схемы, 
иллюстрирующей взаимосвязь между узлами сети, а также определение компонентов сети, их связей, а также источников и приемников информационных потоков.
\item Конфигурация параметров для имитационного моделирования: настройка параметров моделирования, к которым относятся настройка сетевых протоколов, 
установка размеров буферов, временных задержек, пропускной способности каналов связи и других сетевых характеристик.
\item Разработка сценария имитационного моделирования: создание сценария, описывающего последовательность событий в сети, включая передачу данных между узлами, 
модификации в структуре сети, адаптацию параметров сетевых протоколов и прочее.
\item Инициация процесса моделирования: запуск процесса моделирования, в ходе которого NS-2 выполняет воспроизведение передачи данных через сеть, реагируя на 
события согласно установленным параметрам и сценарию.
\item Сбор данных и их анализ по завершении моделирования: В процессе имитационного моделирования NS-2 аккумулирует информацию о показателях работы сети, 
включая задержки, пропускную способность, частоту потерь пакетов и другие важные метрики.
\end{enumerate}

\section{Mininet}
\label{chap1:sec2}

Mininet ~\cite{mininet} — это симулятор сетевых топологий на основе виртуаилизации,
который позволяет моделировать и изучать поведение сетей в
контролируемой среде, основанный на использовании виртуальных машин и
пространств имен Linux для создания изолированных сетевых
узлов. Моделирование сетевых топологий с помощью Mininet позволяет
исследовать различные сетевые протоколы, маршрутизацию, управление
трафиком и т.д. Возможности моделирования с помощью Mininet включают
создание виртуальных сетевых узлов, конфигурирование топологий (связь
между узлами, настраивать IP-адреса, маршрутизацию), имитировать
различные условия сети, такие как задержки, потери пакетов и
пропускную способность, интеграция с контроллерами для исследования
новых протоколов и алгоритмов. 

В архитектуре Mininet основные элементы включают в себя виртуальные хосты, 
которые функционируют как компьютеры с возможностью запуска собственных 
процессов и сетевых настроек. Для моделирования сетевых коммутаторов 
используются виртуальные коммутаторы, при этом часто применяется Open vSwitch для 
поддержки программно-конфигурируемой сетевой среды через OpenFlow. Взаимосвязь между 
хостами и коммутаторами осуществляется через виртуальные соединения, имитирующие физические 
кабельные подключения с помощью виртуальных Ethernet-интерфейсов. Контроль над сетевым трафиком 
обеспечивается через контроллеров, которые могут быть как встроенными, так и подключаться внешними 
устройствами, облегчая тестирование различных стратегий управления сетью. Такая структура позволяет 
Mininet обеспечивать высокую степень изоляции и контроля над сетевым окружением, используя при этом 
пространства имен и виртуальные интерфейсы для создания масштабируемых и гибких сетевых топологий 
на одном компьютере или сервере.

Алгоритм работы в Mininet включает в себя несколько ключевых шагов:
\begin{enumerate}
\item Подготовка виртуальной сетевой инфраструктуры: создание виртуальной сетевой топологии с использованием пространств имен Linux и технологии виртуализации, позволяющее моделировать работу отдельных сетевых узлов, как коммутаторов, так и конечных устройств, связывая их виртуальными каналами связи с заданными параметрами пропускной способности, задержки и потерь.
\item Настройка параметров сетевой топологии: конфигурация созданной виртуальной сети, включая назначение IP-адресов узлам, определение маршрутов передачи данных и установление правил обработки трафика на коммутаторах.
\item Эмуляция сетевых условий: имитация различных условий работы сети, таких как изменение пропускной способности каналов, введение искусственных задержек и эмуляция потерь пакетов, что позволяет оценить поведение сетевых протоколов и приложений в разнообразных сценариях, включая условия высокой загруженности сети и ненадежности каналов связи.
\item Запуск и мониторинг симуляции: запуск симуляции в Mininet, в процессе которого можно осуществлять мониторинг состояния сети, отслеживая ключевые метрики производительности.
\item Анализ полученных данных: завершающий этап работы с Mininet включающая в себя сбор и анализ данных, полученных в ходе симуляции, что позволяет оценить эффективность сетевых протоколов, алгоритмов маршрутизации и стратегий управления трафиком, а также верифицировать теоретические модели работы сети на практике.
\end{enumerate}

Разберем сторонние программные средства, которые можно использовать совместно с mininet

\subsection{Iperf3}

iPerf3~\cite{iperf3} --- это кроссплатформенное клиент-серверное приложение с открытым исходным кодом, 
разработанный для оценки пропускной способности сети между двумя устройствами в сети. 
Данный инструмент позволяет проводить тестирование с использованием различных транспортных протоколов, включая TCP, UDP и SCTP, 
что обеспечивает гибкость при анализе сетевых характеристик в разнообразных условиях.

Для протоколов TCP и SCTP iPerf3 предлагает следующие функции:
\begin{itemize}
\item Оценка пропускной способности, что позволяет измерить максимальную скорость передачи данных между двумя узлами.
\item Возможность настройки размера MSS и MTU, обеспечивая 
    тем самым способ оптимизации и адаптации сетевого взаимодействия под специфические условия.
\item Мониторинг размера окна перегрузки TCP, что важно для анализа поведения TCP при управлении перегрузками в сети.
\end{itemize}

При работе с UDP, iPerf3 предоставляет следующие возможности:
\begin{itemize}
\item Измерение пропускной способности для определения максимальной скорости передачи данных без установления соединения.
\item Оценка потерь пакетов, что критически важно для приложений, чувствительных к потерям, например, для голосовых и видео приложений.
\item Измерение колебания задержки, позволяющее оценить стабильность сетевой задержки, 
    что особенно актуально для реального времени мультимедийных приложений.
\item Поддержка групповой рассылки пакетов, расширяющая возможности 
    тестирования сети за счет эмуляции трансляций для множества получателей.
\end{itemize}

Основные парметры iperf3 представлены в таблице ~\ref{tab1}

\begin{table}[!h]
\caption{Основные параметры iperf3}
\label{tab1}
{\footnotesize{ 
\begin{tabular}{|p{0.2\textwidth}|p{0.6\textwidth}|}
  \hlx{hv}
  Параметр & Значение \\ 
  \hlx{vh}
  -c <hostname/IP> & запуск Iperf в режиме клиента, указывая адрес сервера (hostname или IP), с которым следует соединиться\\
  \hlx{vh}
  -s & запуск Iperf в режиме сервера, ожидая входящих соединений от клиента\\
  \hlx{vh}
  -D & запуск сервера в фоновом режиме\\
  \hlx{vh}
  -p <port> & назначение порта, который будет использоваться для соединения, по умолчанию 5201\\
  \hlx{vh}
  -t <seconds> &  продолжительность теста в секундах, по умолчанию — 10 секунд\\
  \hlx{vh}
  -J & вывод результатов теста в формат .json\\
  \hlx{vh}
  -1 & завершение работы сервера после передачи данных\\
  \hlx{vh}
  -i <seconds> & интервал между периодическими отчётами о пропускной способности; например, -i 1 будет выводить отчёт каждую секунду\\
  \hlx{vh}
\end{tabular}
}}
\end{table}

Таким образом, iPerf3 представляет собой мощный инструмент для комплексного анализа производительности сети, 
позволяя исследователям и инженерам тонко настраивать и оценивать характеристики сетевой инфраструктуры.
 
\subsection{tc}

Linux Traffic Control (tс) ~\cite{tc}  из пакета iproute2 представляет 
собой мощный инструмент для управления трафиком в операционных системах 
на базе Linux. Она позволяет контролировать поток данных через сетевые 
интерфейсы, обеспечивая возможность управления пропускной способностью, 
задержками, потерями пакетов и другими параметрами сетевого трафика.
Позволяет получить данные осостоянии интерфейса

Ключевые компоненты
\begin{itemize}
\item qdisc: основной компонент для контроля над трафиком. Определяет, как система управляет и отправляет пакеты. Существует несколько типов qdisc, каждый из которых предоставляет различные механизмы управления очередями.
\item class: используется вместе с некоторыми типами qdisc (например, HTB или Hierarchy Token Bucket) для создания иерархии и дополнительного управления пропускной способностью.
\item filter: фильтры применяются для классификации трафика и его распределения между различными классами или qdisc.
\end{itemize}

%%% Local Variables:
%%% mode: latex
%%% coding: utf-8-unix
%%% TeX-master: "../default"
%%% End:
