%% Приложение
\chapter{Натурная модель}

В данном приложении представлен исходный код программы
для решения стохастических дифференциальных уравнений,
написанный на языке C с использоваеним библиотеки GSL.

\section*{topology.py}
\begin{minted}[linenos,tabsize=2,breaklines]{python}
from mininet.topo import Topo
from router import LinuxRouter

# создание топологии для сети
class NetworkTopo(Topo):
    def build(self, num_hosts_subnets, **_opts):
        # создание маршрутизаторы
        r0 = self.addHost('r0', cls=LinuxRouter, ip='10.0.0.1/24')
        r1 = self.addHost('r1', cls=LinuxRouter, ip='10.1.0.1/24')

        # добавление коммутаторов для связи хостов
        s1 = self.addSwitch('s1')
        s2 = self.addSwitch('s2')

        # соединение между маршрутизаторами и коммутаторами
        self.addLink(s1, r0, intfName2='r0-eth1', params2={'ip': '10.0.0.1/24'}, max_queue_size=300)
        self.addLink(s2, r1, intfName2='r1-eth1', params2={'ip': '10.1.0.1/24'}, max_queue_size=300)

        # соединение между маршрутизаторами
        self.addLink(r0, r1, intfName1='r0-eth2', intfName2='r1-eth2', max_queue_size=30000, params1={'ip': '10.100.0.1/24', 'bw': '20'}, params2={'ip': '10.100.0.2/24', 'bw': '15'})

        # добавление оконечных устройств
        for i in range(1, 2*num_hosts_subnets + 1):
            host_ip = '10.0.0.{}/24' if i % 2 == 1 else '10.1.0.{}/24'
            default_route = 'via 10.0.0.1' if i % 2 == 1 else 'via 10.1.0.1'
            host = self.addHost(name='h{}'.format(i), ip=host_ip.format(250 - i), defaultRoute=default_route)
            switch = s1 if i % 2 == 1 else s2
            self.addLink(host, switch, bw='100m')
\end{minted}

\section*{utils.py}
\begin{minted}[linenos,tabsize=2,breaklines]{python}
from threading import Thread
import threading
import time

iperf_time = 101
# Функция для запуска iperf между парой хостов
def run_iperf(net, host1_name, host2_name):
    
    h1 = net.get(host1_name)
    h2 = net.get(host2_name)
    h2.cmd("iperf3 -s -D")
    #time.sleep(5)
    h1.cmd(f"mkdir -p output/{host1_name}_to_{host2_name}")
    h1.cmd(f'iperf3 -c {h2.IP()} -w 40000 -t {iperf_time} -l 1000 -J > output/{host1_name}_to_{host2_name}/{host1_name}_to_{host2_name}_iperf3.json')
    h1.cmd(f"cd output/{host1_name}_to_{host2_name} && plot_iperf.sh {host1_name}_to_{host2_name}_iperf3.json")
    
# Функция для запуска мониторинга на линке с RED
def monitor_queue(net, interface="r0-eth2", interval=0.025, output_file='queue_monitor.txt'):
    t = threading.current_thread()
    with open(output_file, 'w') as file:
        while getattr(t, "do_run", True):
            result = net['r0'].cmd(f'tc -s qdisc show dev {interface}')
            print(result, file=file)
            #time.sleep(interval)

\end{minted}
%%% Local Variables:
%%% mode: latex
%%% coding: utf-8-unix
%%% TeX-master: "../default"
%%% End:
