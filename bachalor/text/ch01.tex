% Вторая глава работы 
\chapter{Обзор дисциплины управлением очередью RED}
\label{chap2}

В данном разделе представим описание работы нескольких алгоритмов семейства RED и их реализацию в NS-2 и mininet. 

\section{Классическая модификация}
\label{chap2:sec1}

RED — это семейство механизмов предотвращения перегрузки
на шлюзе. Он основан на общих принципах, полезен для управления
средним размером очереди в сети, где не доверяют взаимодействию между
протоколами передачи данных. В контрасте с подходом Droptail, который предусматривает 
простое отбрасывание входящих пакетов при достижении максимальной емкости очереди, RED учитывает
потоки трафика в сети и стремится предоставить равную пропускную
способность для каждого соединения, что позволяет избежать перегрузки
сети и улучшить качество обслуживания. В оригинальном RED
маршрутизатор вычисляет усредненный по времени средний размер очереди
с использованием фильтра нижних частот (экспоненциально взвешенное
скользящее среднее) или сглаживания по длине выборки очередей, средний
размер очереди сравнивается с двумя пороговыми значениями: минимальным
порогом и максимальным. Когда средний размер очереди меньше
минимального порога, пакеты не отбрасываются, когда средний размер
очереди превышает максимальный порог, отбрасывается все поступающие
пакеты. Если размер средней очереди находится между минимальным и
максимальным порогом, пакеты отбрасываются с вероятностью $p$, которая
линейно увеличивается до тех пор, пока средняя очередь не достигнет
максимального порога. Подробно классический алгоритм описан в~\cite{RED1, RED0}.


На рисунке ~\ref{fig1} представлена вероятная функция сброса пакетов.
 
\begin{figure}[h!]
 \centerline{\includegraphics[width=0.7\textwidth]{red_plot}}
 \caption{Вид функции сброса в алгоритме RED}
\label{fig1}
\end{figure}


Вероятность $p_{b}$ маркировки на отбрасывание пакетов представляет
собой функцию, линейно зависящую от $\hat{q}$ (средневзвешенное 
скользящее среднее), минимального $q_{\min}$ и максимального
$q_{\max}$ пороговых значений и параметра $p_{\max}$, определяющего
часть отбрасываемых пакетов при достижении средним размером очереди
значения $q_{\max}$ и вычисляется следующим образом \eqref{eq2:1}:

\begin{equation}
\label{eq2:1}
p_{b} = \begin{cases}
        0, &  \ 0 < \hat{q} \leqslant q_{\min},
        \\
        \frac{\hat{q} - q_{\min}}{q_{\max} - q_{\min}} p_{\max}, & \ q_{\min} < \hat{q} \leqslant q_{\max}, 
        \\
        1, &  \ \hat{q} > q_{\max}.
\end{cases}                                     
\end{equation}


В NS-2 файлы, связанные с RED, прописаны в каталоге
\verb|ns-2.35/queue|, где представлены также другие реализации
очередей (среди них DropTail, BLUE и т.д.). Следует уделить внимание
двум файлам: \verb|red.cc| (исходники), и \verb|red.h| (заголовочный
файл). Вероятность отбрасывания пакета прописана в функции

\verb|double REDQueue::calculate_p_ne файла red.cc|

Для реализации в NS-2 необходимо указать в качестве очередей между соединениями
RED, и при настройке очереди указать минимальные и максимальные пороговые значения 
\verb|(thresh_ и maxthresh_)|, величина, обратное параметру максимального сброса\verb|(linterm_)|, 
а также указать параметр \verb|gentle_| false. 


Для реализации в Mininet используем утилиту tc qdisk ... red, имеющий следующие опции:
\begin{itemize}
\item min: минимальный порог, по достижении которого возникает вероятность отметки пакета.
\item max: максимальный порог очереди
\item probability: максимальная вероятность пометки, указанная как число с плавающей точкой, от 0.0 до 1.0. 
\item limit: жесткий предел реального (не среднего) размера очереди в байтах. По достижении этого размера все лишние пакеты будут отброшены.
\item burst: используется для определения того, как реальный размер очереди начинает влиять на средний размер очереди. 
\item avpkt: указывается в байтах. Используется вместе с burst для определения временной константы для вычисления среднего размера очереди.
\item bandwidth: используется для расчета среднего размера очереди после простоя в течение некоторого времени. Должно быть равным значению пропускной способности интерфейса. Не влияет на параметр пропускной скорости интерфейса. Необязательное значение.

\end{itemize}

Существует несколько причин, по которым существует множество вариаций алгоритмов семейства RED:

\begin{enumerate}
\item Разнообразные сетевые сценарии. Разные сетевые сценарии требуют разных настроек и параметров для эффективного управления потоком. Например, алгоритм RED может быть настроен по-разному для использования в локальной сети (LAN) и в глобальной сети (WAN) или в сетях с разной пропускной способностью.
\item Разные типы сетей. RED может быть применен в разных типах сетей, включая проводные и беспроводные сети, и разные типы сетей могут иметь уникальные характеристики и требования, которые влияют на алгоритм.
\item Эволюция сетевых технологий. Сетевые технологии постоянно развиваются, и новые требования и возможности могут потребовать адаптации алгоритма RED. Например, изменения в сетевых протоколах или появление новых типов трафика могут потребовать модификаций алгоритма RED.
\item Эксперименты и исследования. Сетевые исследователи могут создавать различные вариации RED для проведения экспериментов и оценки их производительности в различных условиях.
\item Открытая архитектура. RED - это открытая архитектура, что позволяет исследователям и инженерам создавать свои собственные модификации и адаптации алгоритма в соответствии с конкретными потребностями и задачами.
\end{enumerate}


\section{Нелинейные и кусочно-линейные модификации}
\label{chap2:sec2}

\subsection{NLRED}

Nonlinear RED~--- это модификация классического алгоритма RED, в
котором используется нелинейная функция для определения
вероятности отбрасывания пакетов. 
 
Вероятность $p_{b}$ маркировки на
отбрасывание пакетов вычисляется следующим способом \eqref{nlred}:

\begin{equation}
\label{nlred}
p_{b} = \begin{cases}
        0, &  \ 0 < \hat{q} \leqslant q_{\min},
        \\
        1.5({\frac{\hat{q} - q_{\min}}{q_{\max} - q_{\min}})^2} {p_{\max}}, & \ q_{\min} < \hat{q} \leqslant q_{\max},
        \\
        1, &  \ \hat{q} > q_{\max}.
\end{cases}
\end{equation}

Замена линейной функции вероятности сброса пакетов на нелинейную квадратичную функцию поспосодствовала тому, что,
унаследовав простоту RED, NLRED менее чувствителен к настройкам параметров, имеет более предсказуемый средний размер очереди 
и может достичь более высокой пропускной способности. Nonlinear RED предназначен для более точной адаптации к изменениям трафика и динамике сети. 
Он способен эффективно реагировать на изменения величины очереди и адаптироваться к различным условиям сети. 
Это позволяет более гибко управлять задержкой пакетов и предотвращать перегрузки в сети, 
что делает Nonlinear RED более эффективным по сравнению с классическим алгоритмом RED.~\cite{NLRED1}.  



По умолчанию NLRED не реализован в NS-2. Для её добавления я использовал патч для данной модификации, созданный Mohit
  P. Tahiliani для версии 2.34, совместимой также для версии 2.35 ~\cite{nlredpatch}. 
  
\begin{enumerate}
\item Установил к себе на машину патч \verb|NLRED.patch| от 
\item Отредактировал файл, заменив везде номер версии на 2.35 и переместил в каталог \verb|ns-allinone|.
\item Дополнил файлы \verb|queue/red.cc|, \verb|queue/red.h|, \verb|tcl/ns-default.tcl| строками из патча, .
\item Переустановил программу.
\item В настройке очереди сети указал значение переменной \verb|nonlinear_ 1|.
\end{enumerate}

\subsection{GRED}

GRED ~--- алгоритм активного управления очередью,
является одним из основных расширений RED. Стандартный алгоритм увеличивает
вероятность отбрасывания с 0.05 до 0.5, когда средняя длина очереди
увеличивается от минимального до максимального порогового значения, но
при превышении максимального порога вероятность возрастает напрямую с
0.5 до 1.  Этот внезапный скачок нормализуется модификацией Gentle
RED, который расширяет RED тем, что добавляет дополнительное
максимальное пороговое значние, которое равно $2q_{\max}$, тем самым
<<сглаживая>> кривую~\cite{GRED}. Однако, например, задача минимального порога в данной модификации не меняется, 
и увеличение лишь максимального порога для отбрасывания всех пакетов делает GRED лишь частным случаем классического алгоритма. 
Данная модификация в NS-2 используется по умолчанию, так как переменная \verb|gentle_| по умолчанию 
является истинной. 

Вероятность сброса определяется по формуле \eqref{gred}:

\begin{equation}
\label{gred}
p_{b} =\begin{cases}
        0, &  \  0 < \hat{q} \leqslant q_{\min}, 
        \\
        \frac{\hat{q} - q_{\min}}{q_{\max} - q_{\min}} p_{\max}, & \ q_{\min} \leqslant \hat{q} < q_{\max}, 
        \\
        \frac{\hat{q} - q_{\min}}{q_{\max}}(1-p_{\max}) - p_{\max}, & \ q_{\max} \leqslant \hat{q} < 2q_{\max}, 
        \\
        1, &  \ \hat{q} \geqslant  q_{\max}.
\end{cases}
\end{equation}


\subsection{DS-RED}

Алгоритм DS-RED ~--- это ещё одна модификация RED, в котором вводится дополнительное пороговое значение $q_{mid}$ между минимальным $q_{min}$ и максимальным
REDФункция сброса описывается двумя линейными сегментами с углами наклона $\alpha $ и $\beta $ соответственно, регулируемыми
задаваемым селектором режимов $\gamma$ ~\cite{DSRED}. 

Функция вероятности сброса пакетов в алгоритме DSRED показана в формуле \eqref{dsred}

\begin{equation}
\label{dsred}
p_{b} =\begin{cases}
        0, &  \  0 < \hat{q} \leqslant q_{min}, 
        \\
        \alpha{\hat{q} - q_{min}}, & \ q_{min} \leqslant \hat{q} < q_{mid}, 
        \\
        1 - \gamma + \beta{\hat{q} - q_{mid}}, & \ q_{mid} \leqslant \hat{q} < q_{max}, 
        \\
        1, &  \ \hat{q} \geqslant  q_{max}.
\end{cases}
\end{equation}

где $\alpha = (\frac{2(1 - \gamma)}{\hat{q} - q_{\min}})$, а $\beta = (\frac{2\gamma}{\hat{q} - q_{min}})$

Режим работы DSRED настраивается с помощью изменения наклона функции выброса DSRED с использованием параметра $\gamma$. 
Регулируя только $\gamma$, DSRED способен достичь высокой скорости выброса, за которой следует низкая скорость выброса, 
или наоборот, обеспечивая гибкую схему управления для решения сложных ситуаций с сетевой конгестией.

DSRED похож на RED в двух аспектах. Во-первых, оба алгоритма используют линейные функции сброса для обеспечения плавно 
увеличивающегося действия сброса на основе средней длины очереди. Во-вторых, они используют ту же функцию для расчета 
средней длины очереди. В результате вышеуказанных двух сходств DSRED наследует преимущества RED. 
Однако двухсегментная функция сброса DSRED обеспечивает гораздо более гибкую операцию сброса, чем RED.
Двухсегментная функция сброса DSRED использует среднюю длину очереди, которая связана с уровнем конгестии в долгосрочной перспективе. 
С увеличением конгестии вероятность сброса пакетов будет увеличиваться с более высокой скоростью, а не с постоянной скоростью (как в RED). 
Это даст раннее предупреждение хостам о необходимости снижения нагрузки, предотвращая ухудшение конгестии.

По умолчанию модификация не реализована в NS-2, для её реализации дополнил фунцию \verb|double REDQueue::calculate_p_ne файла red.cc|, 
а в программе очереди указал значение переменной \verb|double_slope_ 1|. 


\subsection{HRED}

HRED ~---это модификация классического RED с нилейно возрастающей функцией отбрасывания пакетов в сети.
Использование HRED гиперболы в качестве кривой вероятности выброса может регулировать размер очереди к $q_{\max}$ 
в гораздо более широком диапазоне нагрузок на трафик. Другими словами, HRED нечувствителен к уровню сетевой нагрузки, 
в результате чего задержка в очереди становится более предсказуемой, поскольку размер очереди не изменяется сильно в 
зависимости от уровня конгестии. HRED сохраняет способность контролировать кратковременную перегрузку путем поглощения 
пакетных потоков, так как он все еще использует алгоритм подсчета среднего размера очереди и поддерживает неполную очередь. 
HRED прост в реализации и легко внедряется на маршрутизаторах, так как зменяется только профиль отбрасывания по сравнению с 
классическим алгоритмом RED~\cite{HRED}. Для его реализации дополнил фунцию \verb|double REDQueue::calculate_p_ne файла red.cc|, 
а в программе очереди указал значение переменной \verb|hyperbola_ 1|.
 
Вероятность $p_{b}$ маркировки на
отбрасывание пакетов вычисляется следующим способом \eqref{hred}:

\begin{equation}
\label{hred}
p_{b} = \begin{cases}
        0, &  \ 0 < \hat{q} \leqslant q_{\min},
        \\
        1.5({\frac{\hat{q} - q_{\min}}{q_{\max} - q_{\min}})^{-1}} {p_{\max}}, & \ q_{\min} < \hat{q} \leqslant q_{\max},
        \\
        1, &  \ \hat{q} > q_{\max}.
\end{cases}
\end{equation}


\subsection{TRED}

TRED(Three-section random early detection) ~--- это разновидность алгоритма RED, основанный на Nonlinear RED, 
которая направлена на решение проблем недостаточного использования пропускной способности и больших задержек, 
возникающих при низкой и высокой нагрузке в RED. Средняя длина очереди TRED между двумя пороговыми значениями 
разделена на три равные секции, и вероятность отбрасывания пакетов для каждой секции устанавливается по-разному, 
чтобы адаптироваться к различным трафиковым нагрузкам. С использованием симуляции в среде NS2, TRED эффективно 
устраняет недостатки RED, увеличивая пропускную способность при низкой нагрузке и снижая задержку при высокой нагрузке. 
TRED улучшает способность регулировать сетевую перегрузку, повышая использование ресурсов сети и стабильность схемы ~\cite{TRED}. 
По умолчанию модификация не реализована в NS-2, для её реализации дополнил фунцию \verb|double REDQueue::calculate_p_ne файла red.cc|, 
а в программе очереди указал значение переменной \verb|three_sections_ 1|. 

Вероятность $p_{b}$ маркировки на отбрасывание приведена в \eqref{TRED}, где $ \delta = (q_{max} - q_{min})/3 $.

\begin{equation}
\label{TRED}
p_{b} = \begin{cases}
        0, &  \ 0 \leqslant \hat{q} < q_{\min},
        \\
        9({\frac{\hat{q} - q_{\min}}{q_{\max} - q_{\min}})^3} {p_{\max}}, & \ q_{\min} \leqslant  \hat{q} < q_{\min + \delta},
        \\
        (\frac{\hat{q} - q_{\min}}{q_{\max} - q_{\min}}) {p_{\max}}, & \ q_{\min} + \delta \leqslant \hat{q} < q_{\min + 2\delta},
        \\
        9({\frac{\hat{q} - q_{\min}}{q_{\max} - q_{\min}})^3} {p_{\max}} + {p_{\max}}, & \ q_{\min} +2\delta \leqslant  \hat{q} < q_{\max},
        \\
        1, &  \ \hat{q} \geqslant q_{\max}.
\end{cases}
\end{equation} 


\subsection{RED-QL}

RED-QL ~---модификация алгоритма RED, также является разновидностью алгоритма с нелинейно возрастающей функцией. 
RED-QL имеет квадратично-линейную форму и определяется на основе параметров, 
которые могут быть настроены для определенных требований сети\cite{REDQL}. Дополнительный параметр, 
используемый в QRED, способствует повышению его эффективности при большом количестве источников для передачи данных, уменьшая задержку,
потерю пакетов при одинаковой пропускной способности.   
По умолчанию модификация не реализована в NS-2, для её реализации дополнил фунцию 
\verb|double REDQueue::calculate_p_ne файла red.cc|, а в программе очереди указал значение переменной \verb|quadratic_linear_ 1|. . 

Вероятность $p_{b}$ маркировки на отбрасывание приведена в \eqref{RED-QL}, где $ Target = 2(q_{max} + q_{min})/3 - q_{min} $.

\begin{equation}
\label{RED-QL}
p_{b} = \begin{cases}
        0, &  \ 0 \leqslant \hat{q} < q_{\min},
        \\
        9({\frac{\hat{q} - q_{min}}{2(q_{\max} - 2q_{\min}})^2} {p_{\max}}, &  q_{\min} \leqslant  \hat{q} < {Target},
        \\
        p_{max} + 3(1-p_{max}) (\frac{\hat{q} - Target}{q_{\max} + q_{\min}}), & {Target} \leqslant  \hat{q} < q_{max},
        \\
        1, &  \ \hat{q} \geqslant q_{max}.
\end{cases}
\end{equation}


\subsection{SmRED}

SmRED ~--- модификация RED, в которой
вероятность отбрасывания пакетов регулируется в зависимости от нагрузки трафика для достижения оптимальной сквозной производительности.
Кроме того, переход с RED на SmRED в реальной сети требует очень мало работы из-за своей простоты. SmRED эффективно устраняет недостатки
RED, увеличивает пропускную способность при низкой нагрузке и уменьшает задержку при высокой нагрузке ~\cite{SmRED}. 
По умолчанию модификация не реализована в NS-2, для её реализации дополнил фунцию 
\verb|double REDQueue::calculate_p_ne файла red.cc|, а в программе очереди указал значение переменной \verb|smart_ 1|. 

Вероятность $p_{b}$ маркировки на отбрасывание приведена в \eqref{SmRED}, где $ Target = (q_{max} - q_{min})/2 + q_{min} $.

\begin{equation}
\label{SmRED}
p_{b} = \begin{cases}
        0, &  \ 0 \leqslant \hat{q} < q_{\min},
        \\
        ({\frac{\hat{q} - q_{min}}{q_{\max} - q_{\min}})^2} {p_{\max}}, &  q_{\min} \leqslant  \hat{q} < {Target},
        \\
        \sqrt{{\frac{\hat{q} - q_{min}}{q_{\max} - q_{\min}}}} {p_{\max}}, & {Target} \leqslant  \hat{q} < q_{max},
        \\
        1, &  \ \hat{q} \geqslant q_{max}.
\end{cases}
\end{equation}


\section{Адаптивные модификации}
\label{chap2:sec3}

\subsection{ARED}

В алгоритме ARED функция сброса модифицируется
посредством изменения по принципу AIMD, заключающейся в том, что
увеличение некоторой величины производится путём сложения с некоторым
параметром, у уменьшение~--- путём умножения на
параметр ~\cite{ARED1, ARED2}. 

Если классический RED очень зависит от выбора параметров, то для ARED параметры зависят от условий в сети. 
В традиционном RED выбор подходящих значений параметров для достижения этой цели чрезвычайно сложен. 
При небольшой нагрузке на сеть или при высоком значении параметра $p_{\max}$  среднее значение длины очереди колеблется вокруг минимального порога. 
При высокой нагрузке на сеть или при низком значении параметра $p_{\max}$  среднее значение длины очереди колеблется вокруг максимального порога 
и часто превышает его. Для алгоритма ARED параметр $p_{\max}$ изменяется в процессе работы маршрутизатора таким образом, чтобы 
средняя длина очереди поддерживалась между минимальным и максимальным порогами. 
Данный подход снижает проблему изменчивости в задержках очереди и 
минимизирует количество отброшенных пакетов. 

Алгоритм ARED функционирует следующим образом (\eqref{ad1}),
(\eqref{ad2}). Для каждого интервала \verb|interval| (параметр) в
секундах, если $\hat{q}$ больше целевой (желаемой) $\hat{q_t}$ и
$p_{\max} \leqslant 0,5$, то $p_{\max}$ увеличивается на некоторую
величину $\alpha$; в противном случае, если $\hat{q}$ меньше целевой
$\hat{q_t}$ и $p_{\max}\geqslant 0,01$, то $p_{\max}$ уменьшается в
$\beta$ раз:

\begin{equation}
\label{ad1}
p_{\max} = \left\{
  \begin{aligned}
    & p_{\max}+\alpha, \ \hat{q}>\hat{q_{t}}, \ p_{\max} \leqslant 0,5, \\
    & \beta p_{\max}, \ \hat{q}<\hat{q_{t}}, \ p_{\max} \geqslant 0,01, 
  \end{aligned}
\right.
\end{equation}

\begin{equation}
\label{ad2}
q_{\min}+0,4(q_{\max}-q_{\min}) < \hat{q_t} < q_{\min}+0,6\left(q_{\max}-q_{\min}\right).
\end{equation}


Для реализации модификации в NS-2 необходимо указать в
настройке очереди \verb|set adaptive_ 1|. $\alpha$ и $\beta$ задаются командами \verb|set alpha_| и \verb|set beta_|.     
Для реализации в Mininet нужно указать в tc при настройке RED дополнительно указывается adaptive.


\subsection{RARED}

Алгоритм RARED ~\cite{RARED} является модификаций ARED, который предлагает более активно изменять вероятность сброса $p_{\max}$,
чтобы иметь возможность быстрой адаптации к изменяющейся
экспоненциально взвешенной скользящей средней длине очереди $\hat{q}$. Данная модификация имеет много преимуществ 
над RED и ARED с точки зрения скорости потери пакетов и полезной пропускной способности. 

Функции изменения параметра $p_{\max}$ представлена ниже(\eqref{rf1}), (\eqref{rf2}):

\begin{equation}
\label{rf1}
p_{max} = \left\{
  \begin{aligned}
& p_{\max}+\alpha, \quad  \hat{q}>\hat{q_{t}}, \quad p_{max} \leqslant 0,5, \\
& \beta p_{\max}, \quad \hat{q}\leqslant\hat{q_{t}}, \quad p_{max} > 0,5,
  \end{aligned}
\right.
\end{equation}

\begin{equation}
\label{rf2}
\left\{
  \begin{aligned}
    & q_{\min}+0,48\left(q_{\max}-q_{\min}\right) < \hat{q_t} < q_{\min}+0,52\left(q_{\max}-q_{\min}\right), \\
    & \alpha=\left(0,25\frac{\hat{q}-\hat{q_t}}{\hat{q_t}} \right)p_{\max}, \\ 
    & \beta=1-\left(0,17\frac{\hat{q}-\hat{q_t}}{\hat{q_t}-q_{\min}}\right).
  \end{aligned}
\right.
\end{equation}


По умолчанию RARED не реализован в NS-2. Для её добавления я использовал патч для данной модификации, созданный Mohit
  P. Tahiliani для версии 2.34, совместимой также для версии 2.35 ~\cite{refinedpatch}. 

\begin{enumerate}
\item Установил к себе на машину патч \verb|RARED.patch| от 
\item Отредактировал файл, заменив везде номер версии на 2.35 и переместил в каталог \verb|ns-allinone|.
\item Дополнил файлы \verb|queue/red.cc|, \verb|queue/red.h|, \verb|tcl/ns-default.tcl| строками из патча.
\item Переустановил программу.
\item В настройке очереди указал значение \verb|adaptive_ 1| и 
\verb|refined_adaptive_ 1|.
\end{enumerate}
 

\subsection{POWARED}


Powared является еще одной модификацией алгоритма ARED ~\cite{Powared}. 
В данной модификации величина $p_{max}$ максимального сброса считается следующим образом \eqref{powared}. 

\begin{equation}
\label{powared}
p_{max} =\begin{cases}
        p_{max}-\delta_1, &  \  q_{\min} \leqslant \hat{q} < q_{mid}, 
        \\
        p_{max}+\delta_2, & \ q_{mid} < \hat{q}  \leqslant q_{max}, 
        \\
        p_{max}, &  \ \hat{q} =  q_{mid},
\end{cases}
\end{equation}

где $q_{mid} = 0.5(q_{min} + q_{max})$, 
$\delta_1 = |\frac{((\hat{q} - q_{mid})}{(\beta q_{mid}))}|^K $, а $\delta_2 = |\frac{((q_{mid} - \hat{q})}{(\beta (R -q_{mid})))}|^K.$

Алгоритм POWARED более агрессивно реагирует на изменение cредней очереди, чем ARED. POWARED способен уменьшать как переполнение, 
так и недозаполнение очереди. Он превосходит RED и ARED с точки зрения использования канала и скорости потери пакетов. 
Хотя POWARED довольно нечувствителен к настройкам своих параметров, вопрос оптимизации его настроек остается довольно сложным.

Стратегия, принятая POWARED, заключается в том, чтобы привести средний размер очереди буфера $q_{mid}$ к размеру, 
в котором система находится в точке равновесия рабочего состояния или в устойчивом состоянии.
Когда средний размер очереди сходится к $q_{mid}$, система стабильна, и ее производительность может быть оптимизирована.
POWARED требует лишь простого периодического обновления $p_{max}$ каждые несколько секунд, что аналогично ARED, 
делая накладные расходы малыми и приемлемыми для внедрения в высокоскоростных маршрутизаторах, вводя только два параметра:
фактор степени $k$ и коэффициент сжатия $\beta$.


Данная модификация не реализована в NS-2, для её моделирования я в файл red.cc добавил функцию 
\verb|void REDQueue::updateMaxP_powared|, а в настройке очереди сети указал значение переменных 
\verb|adaptive_ 1| и \verb|powared_ 1|. 
Параметр модификации $k>1$ задается с помощью переменных \verb|pwk_|, по умолчанию $k=2$.



\subsection{FARED}

FARED ~--- это алгортитм, который сохраняет целевой диапазон, указанный в алгоритме RARED, 
но изменяет верхнюю и нижнюю границы для $\alpha $ и $\beta$ соответственно. 
Алгоритм FARED обеспечивает надежную производительность в широком диапазоне сред, 
включая сценарии с умеренной и высокой нагрузкой на трафик~\cite{Tahiliani2012}.
Данная модификация не требует установки каких-либо дополнительных параметров для повышения 
производительности. Поскольку в алгоритм FARED внесены лишь незначительные изменения по сравнению
с ARED и RARED, он может быть развернут без каких-либо сложностей \eqref{fared1}, \eqref{fared2}.


Данная модификация не реализована в NS-2, для её моделирования я в файл red.cc добавил функцию 
\verb|void REDQueue::updateMaxP_fast_adaptive|, а в настройке очереди сети указал значение переменных 
\verb|adaptive_ 1| и \verb|fast_adaptive_ 1|. 

\begin{equation}
\label{fared1}
p_{max} = \left\{
  \begin{aligned}
& p_{\max}+\alpha, \quad  \hat{q}>\hat{q_{t}}, \quad p_{max} \leqslant 0,5, \\
& \beta p_{\max}, \quad \hat{q}\leqslant\hat{q_{t}}, \quad p_{max} > 0,5,
  \end{aligned}
\right.
\end{equation}

\begin{equation}
\label{fared2}
\left\{
  \begin{aligned}
    & q_{\min}+0,48\left(q_{\max}-q_{\min}\right) < \hat{q_t} < q_{\min}+0,52\left(q_{\max}-q_{\min}\right), \\
    & \alpha=\left(0,0412\frac{\hat{q}-\hat{q_t}}{\hat{q_t}} \right)p_{\max}, \\ 
    & \beta=1-\left(0,0385\frac{\hat{q}-\hat{q_t}}{\hat{q_t}-q_{\min}}\right).
  \end{aligned}
\right.
\end{equation}


%%% Local Variables:
%%% mode: latex
%%% coding: utf-8-unix
%%% TeX-master: "../default"
%%% End:
