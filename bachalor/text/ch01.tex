
% Первая глава работы 
\chapter{Средства моделирования сетей передачи данных}
\label{chap1}

В данном разделе представим краткий обзор используемых в работе средств моделирования сетей
передачи данных.

\section{NS-2}
\label{chap1:sec1}

NS-2 — это программное средство моделирования
сетей, использующееся для исследования и анализа поведения
компьютерных сетей.  Запуск имитационной модели в данной среде
позволяет анализировать различные протоколы и алгоритмы сетевой связи.

NS-2 разработан на языке программирования С++ и TCL, имеет открытый исходный код, что обеспечивает
гибкость и расширяемость средства моделирования, например самостоятельную реализацию алгоритма новых дисциплин управления очередью. 
NS-2 содержит библиотеку классов, которые представляют различные элементы сети,
такие как узлы, маршрутизаторы, каналы связи и протоколы передачи
данных. Для создания модели сети определяются характеристики и
параметры каждого элемента сети: пропускная способность канала,
задержки, вероятность потери пакетов и другие. После завершения
симуляции NS-2 предоставляет мощные инструменты анализа результатов,
включая возможность визуализации данных посредством программы NAM, статистический анализ и сравнение результатов
экспериментов, что позволяет изучать и оценивать производительность
различных протоколов и алгоритмов в различных сценариях
сети ~\cite{NS2-1, NS2-2}.

Алгоритм работы в NS-2 включает в себя несколько ключевых шагов:
\begin{enumerate}
\item \textbf{Формирование структуры сетевого взаимодействия:} процесс конструирования структуры сетевого взаимодействия включающая в себя разработку схемы, 
иллюстрирующей взаимосвязь между узлами сети, а также определение компонентов сети, их связей, а также источников и приемников информационных потоков.
\item \textbf{Конфигурация параметров для имитационного моделирования:} настройка параметров моделирования, к которым относятся настройка сетевых протоколов, 
установка размеров буферов, временных задержек, пропускной способности каналов связи и других сетевых характеристик.
\item \textbf{Разработка сценария имитационного моделирования:} создание сценария, описывающего последовательность событий в сети, включая передачу данных между узлами, 
модификации в структуре сети, адаптацию параметров сетевых протоколов и прочее.
\item \textbf{Инициация процесса моделирования:} запуск процесса моделирования, в ходе которого NS-2 выполняет воспроизведение передачи данных через сеть, реагируя на 
события согласно установленным параметрам и сценарию.
\item \textbf{Сбор данных и их анализ по завершении моделирования:} В процессе имитационного моделирования NS-2 аккумулирует информацию о показателях работы сети, 
включая задержки, пропускную способность, частоту потерь пакетов и другие важные метрики.
\end{enumerate}

\section{Mininet}
\label{chap1:sec2}

Mininet ~\cite{mininet} — это симулятор сетевых топологий на основе виртуаилизации,
который позволяет моделировать и изучать поведение сетей в
контролируемой среде, основанный на использовании виртуальных машин и
пространств имен Linux для создания изолированных сетевых
узлов. Моделирование сетевых топологий с помощью Mininet позволяет
исследовать различные сетевые протоколы, маршрутизацию, управление
трафиком и т.д. Возможности моделирования с помощью Mininet включают
создание виртуальных сетевых узлов, конфигурирование топологий (связь
между узлами, настраивать IP-адреса, маршрутизацию), имитировать
различные условия сети, такие как задержки, потери пакетов и
пропускную способность, интеграция с контроллерами для исследования
новых протоколов и алгоритмов. 

Некоторые характеристики, которые указали на создание Mininet, включают в себя:

\begin{itemize}
  \item \textbf{Гибкость:} новые топологии и функции могут быть настроены в программном обеспечении с использованием языков программирования и распространенных операционных систем.
  
  \item \textbf{Применимость:} правильно реализованные прототипы должны быть применимы в реальных сетях на базе оборудования без изменений в исходных кодах.
  
  \item \textbf{Интерактивность:} управление и запуск симулированной сети должны происходить в режиме реального времени, как если бы это происходило в реальных сетях.
  
  \item \textbf{Масштабируемость:} среда прототипирования должна масштабироваться до крупных сетей с сотнями или тысячами коммутаторов на одном компьютере.
  
  \item \textbf{Реализм:} поведение прототипа должно соответствовать реальному поведению с высокой степенью уверенности, чтобы приложения и протоколы могли использоваться без изменений в коде.
\end{itemize}

Алгоритм работы в Mininet включает в себя несколько ключевых шагов:
\begin{enumerate}
\item \textbf{Подготовка виртуальной сетевой инфраструктуры:} создание виртуальной сетевой топологии с использованием пространств имен Linux и технологии виртуализации, позволяющее моделировать работу отдельных сетевых узлов, как коммутаторов, так и конечных устройств, связывая их виртуальными каналами связи с заданными параметрами пропускной способности, задержки и потерь.
\item \textbf{Настройка параметров сетевой топологии:} конфигурация созданной виртуальной сети, включая назначение IP-адресов узлам, определение маршрутов передачи данных и установление правил обработки трафика на коммутаторах.
\item \textbf{Эмуляция сетевых условий:} имитация различных условий работы сети, таких как изменение пропускной способности каналов, введение искусственных задержек и эмуляция потерь пакетов, что позволяет оценить поведение сетевых протоколов и приложений в разнообразных сценариях, включая условия высокой загруженности сети и ненадежности каналов связи.
\item \textbf{Запуск и мониторинг симуляции:} запуск симуляции в Mininet, в процессе которого можно осуществлять мониторинг состояния сети, отслеживая ключевые метрики производительности.
\item \textbf{Анализ полученных данных:} завершающий этап работы с Mininet включающая в себя сбор и анализ данных, полученных в ходе симуляции, что позволяет оценить эффективность сетевых протоколов, алгоритмов маршрутизации и стратегий управления трафиком, а также верифицировать теоретические модели работы сети на практике.
\end{enumerate}


%%% Local Variables:
%%% mode: latex
%%% coding: utf-8-unix
%%% TeX-master: "../default"
%%% End:
