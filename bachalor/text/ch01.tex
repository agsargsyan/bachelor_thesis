
% Первая глава работы 
\chapter{Средства моделирования сетей передачи данных}
\label{chap1}

В данном разделе представим краткий обзор средств моделирования сетей
передачи данных.

\section{NS-2}
\label{chap1:sec1}

NS-2 (Network simaulator 2) — это программное средство моделирования
сетей, использующееся для исследования и анализа поведения
компьютерных сетей.  Запуск имитационной модели в данной среде
позволяет анализировать различные протоколы и алгоритмы сетевой связи.

NS-2 разработан на языке программирования С++ и TCL, что обеспечивает
гибкость и расширяемость средства моделирования.  NS-2 содержит
библиотеку классов, которые представляют различные элементы сети,
такие как узлы, маршрутизаторы, каналы связи и протоколы передачи
данных. Для создания модели сети определяются характеристики и
параметры каждого элемента сети: пропускная способность канала,
задержки, вероятность потери пакетов и другие. После завершения
симуляции NS-2 предоставляет мощные инструменты анализа результатов,
включая возможность визуализации данных посредством программы NAM
(Network animator), статистический анализ и сравнение результатов
экспериментов, что позволяет изучать и оценивать производительность
различных протоколов и алгоритмов в различных сценариях
сети ~\cite{NS2-1, NS2-2}.

\section{Mininet}
\label{chap1:sec2}

Mininet ~\cite{mininet} — это симулятор сетевых топологий на основе виртуаилизации,
который позволяет моделировать и изучать поведение сетей в
контролируемой среде, основанный на использовании виртуальных машин и
пространств имен Linux для создания изолированных сетевых
узлов. Моделирование сетевых топологий с помощью Mininet позволяет
исследовать различные сетевые протоколы, маршрутизацию, управление
трафиком и т.д. Возможности моделирования с помощью Mininet включают
создание виртуальных сетевых узлов, конфигурирование топологий (связь
между узлами, настраивать IP-адреса, маршрутизацию), имитировать
различные условия сети, такие как задержки, потери пакетов и
пропускную способность, интеграция с контроллерами для исследования
новых протоколов и алгоритмов. 

Некоторые характеристики, которые указали на создание Mininet, включают в себя:

\begin{itemize}
  \item \textbf{Гибкость:} новые топологии и функции могут быть настроены в программном обеспечении с использованием языков программирования и распространенных операционных систем.
  
  \item \textbf{Применимость:} правильно реализованные прототипы должны быть применимы в реальных сетях на базе оборудования без изменений в исходных кодах.
  
  \item \textbf{Интерактивность:} управление и запуск симулированной сети должны происходить в режиме реального времени, как если бы это происходило в реальных сетях.
  
  \item \textbf{Масштабируемость:} среда прототипирования должна масштабироваться до крупных сетей с сотнями или тысячами коммутаторов на одном компьютере.
  
  \item \textbf{Реализм:} поведение прототипа должно соответствовать реальному поведению с высокой степенью уверенности, чтобы приложения и протоколы могли использоваться без изменений в коде.
  
  \item \textbf{Возможность совместного использования:} созданные прототипы должны быть легко совместно используемыми с другими сотрудниками, которые могут выполнять и модифицировать эксперименты.
\end{itemize}


\section{Cisco Packet Tracer}
\label{chap1:sec3}

Packet Tracer — это программное средство, предоставляемое компанией
Cisco Systems, позволяющей смоделировать, конфигурировать и отлаживать
сетевые сценарии, широко используемое в области сетевых
технологий. Данное программное обеспечение предоставляет виртуальную
среду, которое позволяет создавать сетевые топологии и настраивать
устройства Cisco: маршрутизаторы, коммутаторы и т.д. Графический
интерфейс позволяет соединять устройства, устанавливать параметры
соединений и задавать настройки протоколов. Cisco Packet Tracer
позволяет имитировать передачу данных в сети. Пользователи могут
выполнять различные тесты связи, проводить диагностику и мониторинг
сетевых устройств, а также создавать и анализировать журналы событий.

\section{GNS-3}
\label{chap1:sec4}


GNS-3 — это программное средство моделирования сетей, позволяющий
создавать виртуальные сети, состоящие из реальных или виртуальных
устройств, и анализировать их поведение. GNS-3 разработан на языке
программирования Python и основан на эмуляторе динамических узлов
Dynamips, который позволяет запускать реальные образы операционных
систем. В отличие от Packet Tracer, GNS-3 позволяет смоделировать не
только устройства Cisco, но и другие устройства, например, Juniper,
Palo, Alto и другие, что позволяет смоделировать различные типы сетей,
включая центры обработки данных и облачные инфраструктуры. Одной из
главных особенностей GNS-3 является интеграция с виртуальными
машинами, что расширяет возможности моделирования. Появляется
возможность создавать сетевые сценарии, в которых виртуальные машины
выполняют реальные функции, такие как серверы, клиенты, точки доступа
Wi-Fi и т.д. Это позволяет проводить натурное моделирование и
получить более реалистичные результаты в рамках виртуальной среды.


%%% Local Variables:
%%% mode: latex
%%% coding: utf-8-unix
%%% TeX-master: "../default"
%%% End:
