%% Приложение

\chapter{Имитационная модель}
\label{app1}

\section*{main.tcl}
\label{app1:sec1}

\begin{minted}[linenos,tabsize=2,breaklines]{bash}
# новый экземпляр объекта Symulator
set ns [new Simulator]
# трейс файл для nam
set nf [open output/out.nam w]
$ns namtrace-all $nf
# количество источников 
set N 20
# создание узлов и соединений
source "nodes.tcl"
# метрики TCP
source "TCP.tcl"
# настройка очереди		
source "queue.tcl"
# настройка времени моделирования  		
source "timing.tcl" 		
# визуализация в nam
source "nam.tcl"   		
# процедура finish
source "finish.tcl"                                                                         
#запуск программы
$ns run
\end{minted}


\section*{nodes.tcl}
\label{app1:sec2}
\begin{minted}[linenos,tabsize=2,breaklines]{bash}
#маршрутизаторы
set node_(r0) [$ns node]  
set node_(r1) [$ns node]  

#источники и приемники
for {set i 0} {$i < $N} {incr i} {
	set node_(s$i) [$ns node] 		
	set node_(s[expr $N + $i]) [$ns node]	
	}

#связи между маршрутизаторами и другими узлами(размер буфера, время, тип очереди)
for {set i 0} {$i < $N} {incr i} {
	$ns duplex-link $node_(s$i) $node_(r0) 100Mb 20ms DropTail
	$ns duplex-link $node_(s[expr $N + $i]) $node_(r1) 100Mb 20ms DropTail
}

#связи между маршрутизаторами(размер буфера, время, тип очереди)
$ns simplex-link $node_(r0) $node_(r1) 20Mb 15ms RED
$ns simplex-link $node_(r1) $node_(r0) 15Mb 20ms DropTail

# Агенты и приложения
for {set t 0} {$t < $N} {incr t} {
	$ns color $t green
	set tcp($t) [$ns create-connection TCP/Reno $node_(s$t) TCPSink $node_(s[expr $N + $t]) $t]
	$tcp($t) set window_ 32
	$tcp($t) set maxcwnd_ 32
	set ftp($t) [$tcp($t) attach-source FTP]
}
\end{minted}

\section*{TCP.tcl}
\label{app1:sec3}
\begin{minted}[linenos,tabsize=2,breaklines]{bash}
# Мониторинг метрик TCP

#размер окна TCP для всех источников
set windowVsTime [open output/WvsT w]

#размер окна TCP для 1 источника 
set windowVsTime_1 [open output/WvsT_1 w]

#время приема-передачи
set rtt [open output/RTT w]

#отклонение времени приема-передачи 
set rttvar [open output/RTTVAR w]

# Функция для получение данных о метриках
proc plotMetric {tcpSource file metric} {
    global ns
    set time 0.01
    set now [$ns now]
    set value [$tcpSource set $metric]
    puts $file "$now $value"
    $ns at [expr $now+$time] "plotMetric $tcpSource $file $metric"
}
\end{minted}

\section*{queue.tcl}
\label{app1:sec4}
\begin{minted}[linenos,tabsize=2,breaklines]{bash}
#Лимит очереди
$ns queue-limit $node_(r0) $node_(r1) 300
$ns queue-limit $node_(r1) $node_(r0) 300

#настройка параметров RED
set redq [[$ns link $node_(r0) $node_(r1)] queue]
$redq set thresh_ 75 
$redq set maxthresh_ 150
$redq set q_weight_ 0.002
$redq set linterm_ 10
$redq set drop-tail_ true
$redq set gentle_ false
$redq set queue-in-bytes false

#мониторинг параметров длины очереди:
set tchan_ [open output/all.q w]
$redq trace curq_
$redq trace ave_
$redq attach $tchan_

#монитринг соеденения между маршрутизаторами
set qmon [$ns monitor-queue $node_(r0) $node_(r1) [open output/qm.out w]] 
[$ns link $node_(r0) $node_(r1)] queue-sample-timeout

#Для реализации разных модификаций RED, реализовано благодаря изменению исходного кода программы

#$redq set nonlinear_ 1
#$redq set hyperbola_ 1 
#$redq set quadratic_linear_ 1
#$redq set three_sections_ 1
#$redq set exponential_ 1
#$redq set smart_ 1
#$redq set double_slope_ 1

#Группа адаптивных алгоритмов
#$redq set adaptive_ 1
#$redq set feng_adaptive_ 1
#$redq set refined_adaptive_ 1
#$redq set fast_adaptive_ 1
$redq set powared_ 1
\end{minted}

\section*{timing.tcl}
\label{app1:sec5}
\begin{minted}[linenos,tabsize=2,breaklines]{bash}
for {set r 0} {$r < $N} {incr r} {
	$ns at 0.0 "$ftp($r) start"
	$ns at 100.0 "$ftp($r) stop"
	$ns at 1.0 "plotMetric $tcp($r) $windowVsTime cwnd_"
}
$ns at 1.0 "plotMetric $tcp(1) $windowVsTime_1 cwnd_"
$ns at 1.0 "plotMetric $tcp(1) $rtt rtt_"
$ns at 1.0 "plotMetric $tcp(1) $rttvar rttvar_"
$ns at 100.0 "finish"
\end{minted}

\section*{nam.tcl}
\label{app1:sec6}
\begin{minted}[linenos,tabsize=2,breaklines]{bash}
#визуализация цветов, формы, располажения узлов в nam
$node_(r0) color "red"
$node_(r1) color "red"
$node_(r0) label "RED"
$node_(r1) shape "square"
$node_(r0) label "square"

$ns simplex-link-op $node_(r0) $node_(r1) orient right
$ns simplex-link-op $node_(r1) $node_(r0) orient left
$ns simplex-link-op $node_(r0) $node_(r1) queuePos 0
$ns simplex-link-op $node_(r1) $node_(r0) queuePos 0

for {set m 0} {$m < $N} {incr m} {
	$ns duplex-link-op $node_(s$m) $node_(r0) orient right
	$ns duplex-link-op $node_(s[expr $N + $m]) $node_(r1) orient left 
}

for {set i 0} {$i < $N} {incr i} {
	$node_(s$i) color "blue"
	$node_(s$i) label "ftp"

}
\end{minted}

\section*{finish.tcl}
\label{app1:sec7}
\begin{minted}[linenos,tabsize=2,breaklines]{bash}
#Finish procedure
proc finish {} {
   global ns nf
   $ns flush-trace
   close $nf
   global tchan_
   #разделение данных мгновееной очереди и средней очереди
   set awkCode {
      {
	 if ($1 == "Q" && NF>2) {
	    print $2, $3 >> "output/temp.q";
	    set end $2
	 }
	 else if ($1 == "a" && NF>2)
	 print $2, $3 >> "output/temp.a";
      }
   }
   set f [open output/temp.queue w]
   puts $f "TitleText: RED"
   puts $f "Device: Postscript"

   if { [info exists tchan_] } {
      close $tchan_
   }
   exec rm -f output/temp.q output/temp.a 
   exec touch output/temp.a output/temp.q
   exec awk $awkCode output/all.q

   puts $f \"queue
   exec cat output/temp.q >@ $f
   puts $f \n\"ave_queue
   exec cat output/temp.a >@ $f
   close $f
   # вывод графиков в xgraph для быстрого просмотра
   exec xgraph -bb -tk -x time -t "TCPRenoCWND" output/WvsT &
   exec xgraph -bb -tk -x time -t "TCPRenoCWND_1" output/WvsT_1 &
   exec xgraph -bb -tk -x time -t "RTT" output/RTT &
   exec xgraph -bb -tk -x time -t "RTTVAR" output/RTTVAR &
   exec xgraph -bb -tk -x time -y queue output/temp.queue &
   #exec nam output/out.nam &
   exit 0
}
\end{minted}

%%% Local Variables:
%%% mode: latex
%%% coding: utf-8-unix
%%% TeX-master: "../default"
%%% End:
