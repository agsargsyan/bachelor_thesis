
% Третья глава работы 
\chapter{Результаты}
\label{chap3}

\section{Название секции}
\label{chap3:sec1}

Для запуска моделей используем сеть со следующей топологией:

\begin{itemize}
\item $N=20$ TCP-источников, $N$ TCP-приёмников, двух маршрутизаторов $R1$
  и $R2$ между источниками и приёмниками ($N$ — не менее 20);
\item между TCP-источниками и первым маршрутизатором установлены
  дуплексные соединения с пропускной способностью 100 Мбит/с и
  задержкой 20 мс очередью типа DropTail;
\item между TCP-приёмниками и вторым маршрутизатором установлены
  дуплексные соединения с пропускной способностью 100 Мбит/с и
  задержкой 20 мс очередью типа DropTail;
\item между маршрутизаторами установлено симплексное соединение
  ($R1$--$R2$) с пропускной способностью 20 Мбит/с и задержкой 15 мс
  очередью типа RED, размером буфера 300 пакетов; в обратную сторону~---
  симплексное соединение ($R2$--$R1$) с пропускной способностью 15 Мбит/с и
  задержкой 20 мс очередью типа DropTail;
\item данные передаются по протоколу FTP поверх TCPReno;
\item параметры алгоритма RED: $q_{\min}=75$, $q_{\max}=150$, $q_w=0,002$, $p_{\max}=0.1$;
\item максимальный размер TCP-окна 32; размер передаваемого пакета 500
  байт; время моделирования~--- не менее 20 единиц модельного времени.
\end{itemize}


\section{Название секции}
\label{chap3:sec2}

Текст.

\section{Название секции}
\label{chap3:sec3}

Текст.

%%% Local Variables:
%%% mode: latex
%%% coding: utf-8-unix
%%% TeX-master: "../default"
%%% End:
