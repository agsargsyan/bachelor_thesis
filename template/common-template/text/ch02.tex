% Вторая глава работы 
\chapter{Название главы}
\label{chap2}

\section{Название секции}
\label{chap2:sec1}

% Пример оформления таблиц
%

А теперь попробуем сравнить стоимость нашей реализации со стоимостью
обыкновенного дисплейного класса (сервер в обычном ДК используется
только как хранилище информации). Рассмотрим таблицу~\ref{tab3}

\begin{table}[!h]
\caption{Сравнительная стоимость ДК на основе обычных ПК и Х-терминалов}
\label{tab3}
{\footnotesize{ 
\begin{tabular}{|p{0.2\textwidth}|p{0.2\textwidth}|p{0.2\textwidth}|p{0.2\textwidth}|}
  \hlx{hv}
  Тип & Комплектация & Стоимость & Полная стоимость (20 шт)\\ 
  \hlx{vh}
  Стандартный компьютер & Pentium IV, ОЗУ 512, диск 40Гб, видеокарта
  Radeon 8700 & 10000 руб & 200000 руб\\
  \hlx{vh}
  Х-терминал & Pentium II, ОЗУ 128 (можно меньше), диск 1 ГБ (можно
  меньше), видеокарта Radeon 8700 & менее 5000 руб & менее 100000 руб \\
  \hlx{vh}
\end{tabular}
}}
\end{table}

%%

\section{Название секции}
\label{chap2:sec2}

Текст.

\section{Название секции}
\label{chap2:sec3}
 
Текст.


%%% Local Variables:
%%% mode: latex
%%% coding: utf-8-unix
%%% TeX-master: "../default"
%%% End: